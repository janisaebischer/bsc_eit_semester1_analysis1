In der 2. Übung soll man die Funktion:
\begin{equation}
    \label{eq:exercise02-function}
    f(x) = 3x^2
\end{equation}
\noindent In verschiedenen Skalen darstellen. Das Intervall in der Aufgabenstellung wird als $[0, 10]$ angegeben. Da jedoch mit $\log_{10}(x)$ gerechnet wird und 
$\log_{10}(0)$ nicht definiert ist, wird das Intervall auf $(0, 10]$ festgelegt. 

%----------------  Linear - Linear   ---------------------------------------
\subsection{Darstellung des Plots (X-Achse: linear, Y-Achse: linear)}
Als erstes wird die Funktion ganz üblich in der linearen Skala dargestellt. Da die Funktion~\ref{eq:exercise02-function} quadratisch ist, ist im Plot
auch das bekannte Verhalten einer quadratischen Funktion zu sehen. Es ist eine Parabel zu sehen. Wegen dem gewählten Intervall jedoch nur im positiven Spektrum.

\begin{figure}[H]
    \centering
    \includegraphics[width=1\textwidth]{exercise02/aufgabe2_linear_linear_1.png}
    \caption{Plot der Funktion~\ref{eq:exercise02-function} in linearer Skala}
    \label{fig:exercise02-linear-linear}
\end{figure}

%----------------  Log - Linear   ------------------------------------------
\clearpage
\subsection{Darstellung des Plots (X-Achse: logarithmisch, Y-Achse: linear)}
\noindent Die zweite Skala in welcher die Funktion~\ref{eq:exercise02-function} dargestellt werden soll ist die sogenannte semilog Skala.
Eine Achse ist logarithmisch, die vorher quadratische Funktion, sieht jetzt wie eine Exponentialfunktion aus. Es gibt zwei Varianten wie diese Skala erzeugt werden kann.
Die "manuelle" Variante ist das ändern der Skala im Pythonscript. Nach dem umstellen der Skala erhalten wir folgenden Graphen:

\begin{figure}[H]
    \centering
    \includegraphics[width=1\textwidth]{exercise02/aufgabe2_logarithmisch_linear.png}
    \caption{Plot der Funktion~\ref{eq:exercise02-function} in semilog Skala, Python}
    \label{fig:exercise02-semilog-python}
\end{figure}

\clearpage
\paragraph{"Manuelle" Methode} Der gleiche Plot lässt sich auch noch anders realisieren. Dazu wird die gewünschte Achse, in diesem Fall die X-Achse, logarithmisch skaliert.
Die X-Achse wird dazu gleich~\ref{eq:exercise02-u} gesetzt. Die Kurve ist dadurch nicht mehr an gleicher Position wie vorher, die Form jedoch ist die gleiche wie
in Abbildung~\ref{fig:exercise02-semilog-python}.
\begin{equation}
    \label{eq:exercise02-u}
    u = \log_{10}(x)
\end{equation}

\begin{figure}[H]
    \centering
    \includegraphics[width=1\textwidth]{exercise02/aufgabe2_linear_linear_2.png}
    \caption{Plot der Funktion~\ref{eq:exercise02-function} in semilog Skala, u-Methode}
    \label{fig:exercise02-semilog-u-method}
\end{figure}

%----------------  Log - Log   ---------------------------------------------
\clearpage
\subsection{Darstellung des Plots (X-Achse: logarithmisch, Y-Achse: logarithmisch)}
Die dritte Skala in welcher die Funktion~\ref{eq:exercise02-function} dargestellt werden soll ist die logarithmische Skala.
Beide Achsen sind logarithmisch. In der u-v-Darstellung wird die Kurve zu einer Gerade; die Steigung entspricht dem Exponenten 2 und der Achsenabschnitt dem $\log_{10}(3)$.
Es gibt zwei Varianten wie diese Skala erzeugt werden kann. Die "manuelle" Variante ist das ändern der Skala im Pythonscript. 
Nach dem umstellen der Skala erhalten wir folgenden Graphen:

\begin{figure}[H]
    \centering
    \includegraphics[width=1\textwidth]{exercise02/aufgabe2_logarithmisch_logarithmisch.png}
    \caption{Plot der Funktion~\ref{eq:exercise02-function} in logarithmischer Skala, Python}
    \label{fig:exercise02-log-python}
\end{figure}

\clearpage
\paragraph{"Manuelle" Methode} Wie schon im vorherigen Kapitel gesehen, kann auch dies wieder "manuell" über Umrechnung der Skala geschehen. Dafür berechnen wir zusätzlich das $v$.
$u$ und $v$ können anschliessend als Achsen eingetragen werden. Auch hier ist die Skala nicht die selbe wie in ~\ref{fig:exercise02-log-python}, die Form der Kurve
ist jedoch äquivalent.
\begin{equation*}
    u = \log_{10}(x) \text{ und } v = \log_{10}(f(x)) = \log_{10}(y)
\end{equation*}

\begin{figure}[H]
    \centering
    \includegraphics[width=1\textwidth]{exercise02/aufgabe2_linear_linear_3.png}
    \caption{Plot der Funktion~\ref{eq:exercise02-function} in logarithmischer Skala, u-v-Methode}
    \label{fig:exercise02-log-u-v-method}
\end{figure}
