In der zweiten Übung geht es um die Darstellung von Graphen einer Kurve in unterschiedlichen Skalen. Dazu ploten wir jeweils die Funktion
\begin{equation}
    \label{eq:exercise02-function}
    f(x) = 3x^2
\end{equation}
\noindent im Intervall $(0, 10]$. \num{0} wird hier absichtlich ausgeschlossen, da $\log_{10}(0)$ nicht definiert ist.

%----------------  Linear - Linear   ---------------------------------------
\subsection{Darstellung linear-linear}
Als erstes wird die Funktion mit linear skalierter X- und Y-Achse dargestellt. Wie in Abbildung~\ref{fig:exercise02-linear-linear} ersichtlich, führt dies zu einer
quadratischen Funktion.

\begin{figure}[H]
    \centering
    \includegraphics[width=1\textwidth]{exercise02/exercise02-linear-linear.png}
    \caption{Darstellung einer Funktion mit linearer X- und Y-Achse}
    \label{fig:exercise02-linear-linear}
\end{figure}

%----------------  Log - Linear   ------------------------------------------
\clearpage
\subsection{Darstellung log-linear}
\noindent In einem zweiten Schritt wird die Funktion mit logarithmisch skalierter X-Achse dargestellt. In Python kann dazu der Befehl \textit{semilogx()} verwendet werden.
Dieser skaliert die X-Achse logarithmisch ($\log_{10}$), während er die Y-Achse linear belässt. Wie in Abbildung~\ref{fig:exercise02-log-linear-semilogx} ersichtlich, führt
dies zu einer Exponentialfunktion.

\begin{figure}[H]
    \centering
    \includegraphics[width=1\textwidth]{exercise02/exercise02-log-linear-semilogx.png}
    \caption{Darstellung einer Funktion mit logarithmischer X- und linearer Y-Achse (Mit dem Befehl \textit{semilogx()})}
    \label{fig:exercise02-log-linear-semilogx}
\end{figure}

\clearpage
\noindent Die gleiche Kurve lässt sich auch durch \glqq manuelle\grqq~Skalierung der X-Achse erreichen, in dem wir
\begin{equation*}
    u = \log_{10}(x)
\end{equation*}
\noindent setzen und anschliessend auf der Horizontalachse u darstellen. Die Kurve wird dadurch zwar verschoben, es entsteht aber die gleiche Form einer Exponentialkurve
wie mit \textit{semilogx()}.

\begin{figure}[H]
    \centering
    \includegraphics[width=1\textwidth]{exercise02/exercise02-log-linear-plot.png}
    \caption{Darstellung einer Funktion mit logarithmischer X- und linearer Y-Achse (Mit dem Befehl \textit{plot()})}
    \label{fig:exercise02-log-linear-plot}
\end{figure}

%----------------  Log - Log   ---------------------------------------------
\clearpage
\subsection{Darstellung log-log}
Als letztes wird die Funktion mit logarithmisch skalierter X- und Y-Achse dargestellt. In Python kann dazu der Befehl \textit{loglog()} verwendet werden. Dieser
skaliert beide Achsen logarithmisch ($\log_{10}$). Wie in Abbildung~\ref{fig:exercise02-log-log-loglog} ersichtlich, führt dies zu einer linearen Funktion.

\begin{figure}[H]
    \centering
    \includegraphics[width=1\textwidth]{exercise02/exercise02-log-log-loglog.png}
    \caption{Darstellung einer Funktion mit logarithmischer X- und Y-Achse (Mit dem Befehl loglog())}
    \label{fig:exercise02-log-log-loglog}
\end{figure}

\clearpage
\noindent Auch hier lässt sich die gleiche Kurve wieder durch \glqq manuelle\grqq~Skalierung der X- und Y-Achse erreichen, in dem wir
\begin{equation*}
    u = \log_{10}(x) \text{ und } v = \log_{10}(f(x)) = \log_{10}(y)
\end{equation*}
\noindent setzen. Anschliessend tragen wir u auf der horizontalen und v auf der vertikalen Achse auf. Dadurch entsteht ebenfalls eine lineare Funktion.

\begin{figure}[H]
    \centering
    \includegraphics[width=1\textwidth]{exercise02/exercise02-log-log-plot.png}
    \caption{Darstellung einer Funktion mit logarithmischer X- und Y-Achse (Mit dem Befehl plot())}
    \label{fig:exercise02-log-log-plot}
\end{figure}

\noindent Durch Einsetzen und Umformen lässt sich die Funktion dieser Geraden bestimmen:
\begin{align*}
    v &= \log_{10}(f(x)) \\
      &= \log_{10}(3x^2) \\
      &= 2\log_{10}(x) + \log_{10}(3) \\
      &= 2u + \log_{10}(3)
\end{align*}
\noindent Die Gerade hat also eine Steigung von $2u = 2\log_{10}(x)$ und einen Y-Achsenabschnitt bei $y = \log_{10}(3)$.
