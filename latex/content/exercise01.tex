In der 1. Übung soll man die Funktion:
\begin{equation}
    \label{eq:exercise01-function}
    f(x) = \frac{\sin{\frac{1}{x}}}{\frac{1}{x}} \quad\text{für } x > 0
\end{equation}
\noindent in einem Graphen darstellen. Alle wichtigen Parameter dieser Funktion sollen ersichtlich sein.
Aus dem Unterricht haben wir gelernt, dass die Funktion:
\begin{equation}
    \label{eq:exercise01-standard_x/x}
    \lim_{x \to 0} \frac{\sin{x}}{x} = 1
\end{equation}
\noindent gegen den Grenzwert 1 konvergiert. In der Aufgabenstellung ist jedoch eine andere Funktion gegeben.
Diese Funktion kann auch so etwas anders geschrieben werden.
\begin{equation}
    \label{eq:exercise01-standard_(1/x)/(1/x)}
    \lim_{x \to \infty} \frac{\sin{\frac{1}{x}}}{\frac{1}{x}} = 1
\end{equation}
\noindent Aus den gewonnenen Erkenntnissen aus dem Unterricht kann der Grenzwert für die gefragte Funtion~\ref{eq:exercise01-function} berechnet werden.
Da wir wissen, dass der Bruch $1/x, x \to \infty$ extrem klein wird haben wir ein Verhalten des Grenzwerts wie in Funktion~\ref{eq:exercise01-standard_x/x}
ploten wir die verlangte Funtion~\ref{eq:exercise01-function} erhalten wir folgenden Plot: im Intervall $(0, 5]$ ist ersichtlich, dass die Funktion einen Grenzwert bei $y = 1$ hat.

\begin{figure}[H]
    \centering
    \includegraphics[width=1\textwidth]{exercise01/aufgabe1.png}
    \caption{Plot der Funktion~\ref{eq:exercise01-function}}
    \label{fig:exercise01}
\end{figure}

\noindent Zu sehen ist, wie die Funktion gegen 1 konvergiert. Für eine bessere Sichtbarkeit des Grenzwerts, ist eine gestrichelte Asymptote eingezeichnet.
Um den Grenzwert sichtbar und möglichst viel von der Funktion zu zeigen wurde ein Intervall von $(0, 5]$ gewählt. Wichtig! Die Funktion darf nicht durch $0$ geteilt werden.
Dies kann auch so aus der Funktion~\ref{eq:exercise01-function} entnommen werden.
