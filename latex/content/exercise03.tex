In der 3. Übung soll gezeigt werden wie die Funktion:
\begin{equation}
    \label{eq:exercise03-function}
    f(x) = \cos(x)
\end{equation}
im Intervall $(0,\pi)$ numerisch approximiert wird. Die exakte Ableitung der Funktion~\ref{eq:exercise03-function} ist bekannt:
\begin{equation}
    \label{eq:exercise03-prime}
    f'(x) = -\sin(x).
\end{equation}
Um diese Ableitung mit dem Computer zu berechnen, verwenden wir Differenzenquotienten, die den Grenzwert aus der Ableitungsdefinition näherungsweise ersetzen.

\paragraph{Vorwärtsdifferenz.}
Die erste Methode ist der sogenannte Vorwärtsdifferenzenquotient
\begin{equation}
    g(x) = \frac{f(x+h) - f(x)}{h}.
\end{equation}
Dabei wird die Steigung der Sekante durch die beiden Punkte $(x, f(x))$ und $(x+h, f(x+h))$ verwendet, um sich der Tangentensteigung $f'(x)$ zu nähern. 
Für kleinere Werte von $h$ liegen die beiden Punkte näher beieinander, und die Sekante nähert sich der Tangente an.

\paragraph{Zentrale Differenz.}
Die zweite Methode ist der zentrale Differenzenquotient
\begin{equation}
    h(x) = \frac{f\!\left(x+\frac{h}{2}\right) - f\!\left(x-\frac{h}{2}\right)}{h}.
\end{equation}
Hier werden zwei Punkte symmetrisch um $x$ betrachtet, nämlich $x-\frac{h}{2}$ und $x+\frac{h}{2}$. Die Sekante durch diese beiden Punkte schneidet die Kurve 
„links und rechts“ von $x$ und liefert dadurch in vielen Fällen eine bessere Annäherung an die Tangente.

\paragraph{Vergleich der Methoden.}
Damit der Unterschied zwischen den beiden Methoden ersichtlicher wird, ist alles in einem Plot zusammengeführt. Die jeweilig benutzten h-Werte sind im Plot aufgeführt.

\begin{figure}[H]
    \centering
    \includegraphics[width=0.8\textwidth]{exercise03/aufgabe3_3.png}
    \caption{Annäherung der Ableitung der Funktion~\ref{eq:exercise03-function} mit h=3}
    \label{fig:exercise03-1}
\end{figure}

\begin{figure}[H]
    \centering
    \includegraphics[width=0.8\textwidth]{exercise03/aufgabe3_1.png}
    \caption{Annäherung der Ableitung der Funktion~\ref{eq:exercise03-function} mit h=1}
    \label{fig:exercise03-2}
\end{figure}

\begin{figure}[H]
    \centering
    \includegraphics[width=0.8\textwidth]{exercise03/aufgabe3_05.png}
    \caption{Annäherung der Ableitung der Funktion~\ref{eq:exercise03-function} mit h=0.5}
    \label{fig:exercise03-3}
\end{figure}

\begin{figure}[H]
    \centering
    \includegraphics[width=0.8\textwidth]{exercise03/aufgabe3_0001.png}
    \caption{Annäherung der Ableitung der Funktion~\ref{eq:exercise03-function} mit h=0.001}
    \label{fig:exercise03-4}
\end{figure}
