In Übung 3 geht es um die Annäherung der Ableitung. Die Ableitung ist im Allgemeinen wie folgt bestimmt:
\begin{equation*}
    f'(x) = \lim_{h \to 0} \frac{f(x + h) - f(x)}{h}
\end{equation*}
\noindent Mit dieser Formel nähert man sich dem Punkt der Ableitung von nur einer Seite an. Man rechnet also die Steigung der Funktion zwischen dem Punkt, an welchem
man die Ableitung ermitteln möchte und einem zweiten Punkt, welcher um $h$ verschoben ist. Wenn man sich dem genauen Wert der Ableitung an einem Punkt noch schneller
nähern will, kann man folgende Formel verwenden:
\begin{equation*}
    f'(x) = \lim_{h \to 0} \frac{f(x + \frac{h}{2}) - f(x - \frac{h}{2})}{h}
\end{equation*}
\noindent Hiermit rechnet man also die Steigung zwischen zwei Punkten, welche einmal $\frac{h}{2}$ vor dem Punkt der zu ermittelnden Ableitung um einmal $\frac{h}{2}$
dahinter liegen. Da man sich mit dieser Formel der genauen Steigung an einem Punkt \glqq von beiden Seiten\grqq~nähert, erhält man für das gleiche h ein genaueres
Resultat.

%----------------  Derivation with different h   ---------------------------
\subsection{Annäherung der Ableitung von cos(x)}
Um dies an einem praktischen Beispiel zu untersuchen, versuchen wir die Ableitung von $\sin{x}$ durch die oben genannten Formeln zu bestimmen. Die korrekte Ableitung von
ist wie folgt:
\begin{equation*}
    \sin'{x} = \cos{x}
\end{equation*}
Um den Unterschied und die Funktion der beiden Varianten der Annäherung zur Ableitung 
\begin{figure}[H]
    \centering
    \includegraphics[width=0.8\textwidth]{exercise03/exercise03-1.png}
    \caption{Annäherung der Ableitung einer Funktion - Schritt 1: h = 0.8}
    \label{fig:exercise03-1}
\end{figure}

\begin{figure}[H]
    \centering
    \includegraphics[width=0.8\textwidth]{exercise03/exercise03-2.png}
    \caption{Annäherung der Ableitung einer Funktion - Schritt 2: h = 0.4}
    \label{fig:exercise03-2}
\end{figure}

\begin{figure}[H]
    \centering
    \includegraphics[width=0.8\textwidth]{exercise03/exercise03-3.png}
    \caption{Annäherung der Ableitung einer Funktion - Schritt 3: h = 0.2}
    \label{fig:exercise03-3}
\end{figure}

\begin{figure}[H]
    \centering
    \includegraphics[width=0.8\textwidth]{exercise03/exercise03-4.png}
    \caption{Annäherung der Ableitung einer Funktion - Schritt 4: h = 0.1}
    \label{fig:exercise03-4}
\end{figure}

\begin{figure}[H]
    \centering
    \includegraphics[width=0.8\textwidth]{exercise03/exercise03-5.png}
    \caption{Annäherung der Ableitung einer Funktion - Schritt 5: h = 0.01}
    \label{fig:exercise03-5}
\end{figure}