%============================ MAIN DOCUMENT ================================
\PassOptionsToPackage{table}{xcolor}
\documentclass[
  a4paper,
  bibliography=totoc,
  listof=totoc,
  invert-title,
  titleimage-ratio=13
]{bfhpub}

\usepackage[french,english,main=ngerman]{babel}  % https://www.namsu.de/Extra/pakete/Babel.html
\LoadBFHModule{listings,terminal,boxes}

%---------------------------------------------------------------------------
% Documents paths
%---------------------------------------------------------------------------
\makeatletter
\def\input@path{{content/}}
\makeatother

%---------------------------------------------------------------------------
% Hyperref Package (Create links in a pdf)
%---------------------------------------------------------------------------
\usepackage[
	,bookmarks
	,plainpages=false
	,pdfpagelabels
  ,pdfusetitle
	,backref = {false}          % No index backreference
	,colorlinks = {true}        % Color links in a PDF
	,hypertexnames = {true}     % no failures "same page(i)"
	,bookmarksopen = {true}     % opens the bar on the left side
	,bookmarksopenlevel = {0}   % depth of opened bookmarks
	,linkcolor=.
	,filecolor=.
	,urlcolor=.
	,citecolor=.
]{hyperref}

%---------------------------------------------------------------------------
% Base packages
%---------------------------------------------------------------------------
% Include Packages
\usepackage{amsmath}          % various features to facilitate writing math formulas
\usepackage{amsthm}           % enhanced version of latex's newtheorem
\usepackage{amsfonts}         % set of miscellaneous TeX fonts that augment the standard CM
\usepackage{amssymb}          % mathematical special characters

\usepackage{siunitx}

\usepackage{graphicx}         % integration of images
\usepackage{float}            % floating objects

\usepackage{caption}          % for captions of figures and tables
\usepackage{subcaption}       % for subcaptions in subfigures
\usepackage{wrapfig}

\usepackage{exscale}          % mathematical size corresponds to textsize
\usepackage{multirow}         % multirow emables combining rows in tables
\usepackage{multicol}
\usepackage{makecell}

\usepackage{longtable}
\usepackage{adjustbox}

\usepackage[colorinlistoftodos]{todonotes}

\usepackage[table]{xcolor}

\usepackage{array}

\usepackage{mathabx}

\usepackage{attachfile2}

\usepackage{pgfplotstable}
\pgfplotsset{compat=1.18}

\usepackage{totcount}         % Abbildungs- und Tabellenverzeichnis nur bei vorhandenen Einträgen drucken
\newtotcounter{figures}
\newtotcounter{tables}
\AtBeginEnvironment{figure}{\stepcounter{figures}}
\AtBeginEnvironment{table}{\stepcounter{tables}}

%---------------------------------------------------------------------------
% Graphics paths
%---------------------------------------------------------------------------
\graphicspath{{pictures/}{figures/}}

%---------------------------------------------------------------------------
% Blind text -> for dummy text
%---------------------------------------------------------------------------
\usepackage{blindtext}    
\usepackage{letltxmacro}   
\LetLtxMacro{\blindtextblindtext}{\blindtext}

\RenewDocumentCommand{\blindtext}{O{\value{blindtext}}}{
	\begingroup\color{BFH-Gray}\blindtextblindtext[#1]\endgroup
}

%---------------------------------------------------------------------------
% Bibliography Package
%---------------------------------------------------------------------------
\usepackage{csquotes}
\usepackage[backend=biber,style=ieee]{biblatex}
\addbibresource{references.bib}

%---------------------------------------------------------------------------
% Glossary Package
%---------------------------------------------------------------------------
\usepackage[nonumberlist, acronym]{glossaries-extra}
\setabbreviationstyle[acronym]{long-short}
\makeglossaries
%----------------  Glossary Entries  ---------------------------------------
\newglossaryentry{latex}{
    name=\LaTeX,
    description={Ein Textsatzsystem für wissenschaftliche Dokumente}
}

%----------------  Acronyms  -----------------------------------------------


%---------------------------------------------------------------------------
% Makeindex Package
%---------------------------------------------------------------------------
\usepackage{makeidx}
\makeindex

\begin{document}

%------------ START FRONT PART ---------------------------------------------------------------------------------------------------------------------------
\frontmatter % Nummerierung der Seiten in römischen Zahlen

%----------------  BFH tile page   -----------------------------------------
  \title{Projekt 1 mit Python}
  \subtitle{BZG1101a Analysis 1, Herbstsemester 2025/2026\\Bericht}
  \author{Janis Aebischer}
  \department{Technik und Informatik}
  \institute{Elektrotechnik und Informationstechnologie}
  \version{1.0}
  \titlegraphic{\includegraphics{Bild2.png}}
  \partnerlogo{\includegraphics[height=\height]{bfh-logo.pdf}}

  \maketitle

%----------------  Table of contents   -------------------------------------
\clearpage
\tableofcontents

%------------ START MAIN PART ----------------------------------------------------------------------------------------------------------------------------
\mainmatter % Beginn mit normaler Nummerierung der Seiten

%----------------  Introduction   ------------------------------------------
% \clearpage
% \section{Einleitung}
% \label{sec:introduction}
% \input{content/introduction}

%----------------  Exercise 01   -------------------------------------------
\clearpage
\section{Beispiel eines Grenzwerts}
\label{sec:exercise01}
In der 1. Übung soll man die Funktion:
\begin{equation}
    \label{eq:exercise01-function}
    f(x) = \frac{\sin{\frac{1}{x}}}{\frac{1}{x}} \quad\text{für } x > 0
\end{equation}
\noindent in einem Graphen darstellen. Alle wichtigen Parameter dieser Funktion sollen ersichtlich sein.
Aus dem Unterricht haben wir gelernt, dass die Funktion:
\begin{equation}
    \label{eq:exercise01-standard_x/x}
    \lim_{x \to 0} \frac{\sin{x}}{x} = 1
\end{equation}
\noindent gegen den Grenzwert 1 konvergiert. In der Aufgabenstellung ist jedoch eine andere Funktion gegeben.
Diese Funktion kann auch so etwas anders geschrieben werden.
\begin{equation}
    \label{eq:exercise01-standard_(1/x)/(1/x)}
    \lim_{x \to \infty} \frac{\sin{\frac{1}{x}}}{\frac{1}{x}} = 1
\end{equation}
\noindent Aus den gewonnenen Erkenntnissen aus dem Unterricht kann der Grenzwert für die gefragte Funtion~\ref{eq:exercise01-function} berechnet werden.
Da wir wissen, dass der Bruch $1/x, x \to \infty$ extrem klein wird haben wir ein Verhalten des Grenzwerts wie in Funktion~\ref{eq:exercise01-standard_x/x}
ploten wir die verlangte Funtion~\ref{eq:exercise01-function} erhalten wir folgenden Plot: im Intervall $(0, 5]$ ist ersichtlich, dass die Funktion einen Grenzwert bei $y = 1$ hat.

\begin{figure}[H]
    \centering
    \includegraphics[width=1\textwidth]{exercise01/aufgabe1.png}
    \caption{Plot der Funktion~\ref{eq:exercise01-function}}
    \label{fig:exercise01}
\end{figure}

\noindent Zu sehen ist, wie die Funktion gegen 1 konvergiert. Für eine bessere Sichtbarkeit des Grenzwerts, ist eine gestrichelte Asymptote eingezeichnet.
Um den Grenzwert sichtbar und möglichst viel von der Funktion zu zeigen wurde ein Intervall von $(0, 5]$ gewählt. Wichtig! Die Funktion darf nicht durch $0$ geteilt werden.
Dies kann auch so aus der Funktion~\ref{eq:exercise01-function} entnommen werden.


%----------------  Exercise 02   -------------------------------------------
\clearpage
\section{Logarithmische Skala}
\label{sec:exercise02}
In der zweiten Übung geht es um die Darstellung von Graphen einer Kurve in unterschiedlichen Skalen. Dazu ploten wir jeweils die Funktion
\begin{equation}
    \label{eq:exercise02-function}
    f(x) = 3x^2
\end{equation}
\noindent im Intervall $(0, 10]$. \num{0} wird hier absichtlich ausgeschlossen, da $\log_{10}(0)$ nicht definiert ist.

%----------------  Linear - Linear   ---------------------------------------
\subsection{Darstellung linear-linear}
Als erstes wird die Funktion mit linear skalierter X- und Y-Achse dargestellt. Wie in Abbildung~\ref{fig:exercise02-linear-linear} ersichtlich, führt dies zu einer
quadratischen Funktion.

\begin{figure}[H]
    \centering
    \includegraphics[width=1\textwidth]{exercise02/exercise02-linear-linear.png}
    \caption{Darstellung einer Funktion mit linearer X- und Y-Achse}
    \label{fig:exercise02-linear-linear}
\end{figure}

%----------------  Log - Linear   ------------------------------------------
\clearpage
\subsection{Darstellung log-linear}
\noindent In einem zweiten Schritt wird die Funktion mit logarithmisch skalierter X-Achse dargestellt. In Python kann dazu der Befehl \textit{semilogx()} verwendet werden.
Dieser skaliert die X-Achse logarithmisch ($\log_{10}$), während er die Y-Achse linear belässt. Wie in Abbildung~\ref{fig:exercise02-log-linear-semilogx} ersichtlich, führt
dies zu einer Exponentialfunktion.

\begin{figure}[H]
    \centering
    \includegraphics[width=1\textwidth]{exercise02/exercise02-log-linear-semilogx.png}
    \caption{Darstellung einer Funktion mit logarithmischer X- und linearer Y-Achse (Mit dem Befehl \textit{semilogx()})}
    \label{fig:exercise02-log-linear-semilogx}
\end{figure}

\clearpage
\noindent Die gleiche Kurve lässt sich auch durch \glqq manuelle\grqq~Skalierung der X-Achse erreichen, in dem wir
\begin{equation*}
    u = \log_{10}(x)
\end{equation*}
\noindent setzen und anschliessend auf der Horizontalachse u darstellen. Die Kurve wird dadurch zwar verschoben, es entsteht aber die gleiche Form einer Exponentialkurve
wie mit \textit{semilogx()}.

\begin{figure}[H]
    \centering
    \includegraphics[width=1\textwidth]{exercise02/exercise02-log-linear-plot.png}
    \caption{Darstellung einer Funktion mit logarithmischer X- und linearer Y-Achse (Mit dem Befehl \textit{plot()})}
    \label{fig:exercise02-log-linear-plot}
\end{figure}

%----------------  Log - Log   ---------------------------------------------
\clearpage
\subsection{Darstellung log-log}
Als letztes wird die Funktion mit logarithmisch skalierter X- und Y-Achse dargestellt. In Python kann dazu der Befehl \textit{loglog()} verwendet werden. Dieser
skaliert beide Achsen logarithmisch ($\log_{10}$). Wie in Abbildung~\ref{fig:exercise02-log-log-loglog} ersichtlich, führt dies zu einer linearen Funktion.

\begin{figure}[H]
    \centering
    \includegraphics[width=1\textwidth]{exercise02/exercise02-log-log-loglog.png}
    \caption{Darstellung einer Funktion mit logarithmischer X- und Y-Achse (Mit dem Befehl loglog())}
    \label{fig:exercise02-log-log-loglog}
\end{figure}

\clearpage
\noindent Auch hier lässt sich die gleiche Kurve wieder durch \glqq manuelle\grqq~Skalierung der X- und Y-Achse erreichen, in dem wir
\begin{equation*}
    u = \log_{10}(x) \text{ und } v = \log_{10}(f(x)) = \log_{10}(y)
\end{equation*}
\noindent setzen. Anschliessend tragen wir u auf der horizontalen und v auf der vertikalen Achse auf. Dadurch entsteht ebenfalls eine lineare Funktion.

\begin{figure}[H]
    \centering
    \includegraphics[width=1\textwidth]{exercise02/exercise02-log-log-plot.png}
    \caption{Darstellung einer Funktion mit logarithmischer X- und Y-Achse (Mit dem Befehl plot())}
    \label{fig:exercise02-log-log-plot}
\end{figure}

\noindent Durch Einsetzen und Umformen lässt sich die Funktion dieser Geraden bestimmen:
\begin{align*}
    v &= \log_{10}(f(x)) \\
      &= \log_{10}(3x^2) \\
      &= 2\log_{10}(x) + \log_{10}(3) \\
      &= 2u + \log_{10}(3)
\end{align*}
\noindent Die Gerade hat also eine Steigung von $2u = 2\log_{10}(x)$ und einen Y-Achsenabschnitt bei $y = \log_{10}(3)$.


%----------------  Exercise 03   -------------------------------------------
\clearpage
\section{Ableitung}
\label{sec:exercise03}
In Übung 3 geht es um die Annäherung der Ableitung. Die Ableitung ist im Allgemeinen wie folgt bestimmt:
\begin{equation*}
    f'(x) = \lim_{h \to 0} \frac{f(x + h) - f(x)}{h}
\end{equation*}
\noindent Mit dieser Formel nähert man sich dem Punkt der Ableitung von nur einer Seite an. Man rechnet also die Steigung der Funktion zwischen dem Punkt, an welchem
man die Ableitung ermitteln möchte und einem zweiten Punkt, welcher um $h$ verschoben ist. Wenn man sich dem genauen Wert der Ableitung an einem Punkt noch schneller
nähern will, kann man folgende Formel verwenden:
\begin{equation*}
    f'(x) = \lim_{h \to 0} \frac{f(x + \frac{h}{2}) - f(x - \frac{h}{2})}{h}
\end{equation*}
\noindent Hiermit rechnet man also die Steigung zwischen zwei Punkten, welche einmal $\frac{h}{2}$ vor dem Punkt der zu ermittelnden Ableitung um einmal $\frac{h}{2}$
dahinter liegen. Da man sich mit dieser Formel der genauen Steigung an einem Punkt \glqq von beiden Seiten\grqq~nähert, erhält man für das gleiche h ein genaueres
Resultat.

%----------------  Derivation with different h   ---------------------------
\subsection{Annäherung der Ableitung von cos(x)}
Um dies an einem praktischen Beispiel zu untersuchen, versuchen wir die Ableitung von $\sin{x}$ durch die oben genannten Formeln zu bestimmen. Die korrekte Ableitung von
ist wie folgt:
\begin{equation*}
    \sin'{x} = \cos{x}
\end{equation*}
Um den Unterschied und die Funktion der beiden Varianten der Annäherung zur Ableitung 
\begin{figure}[H]
    \centering
    \includegraphics[width=0.8\textwidth]{exercise03/exercise03-1.png}
    \caption{Annäherung der Ableitung einer Funktion - Schritt 1: h = 0.8}
    \label{fig:exercise03-1}
\end{figure}

\begin{figure}[H]
    \centering
    \includegraphics[width=0.8\textwidth]{exercise03/exercise03-2.png}
    \caption{Annäherung der Ableitung einer Funktion - Schritt 2: h = 0.4}
    \label{fig:exercise03-2}
\end{figure}

\begin{figure}[H]
    \centering
    \includegraphics[width=0.8\textwidth]{exercise03/exercise03-3.png}
    \caption{Annäherung der Ableitung einer Funktion - Schritt 3: h = 0.2}
    \label{fig:exercise03-3}
\end{figure}

\begin{figure}[H]
    \centering
    \includegraphics[width=0.8\textwidth]{exercise03/exercise03-4.png}
    \caption{Annäherung der Ableitung einer Funktion - Schritt 4: h = 0.1}
    \label{fig:exercise03-4}
\end{figure}

\begin{figure}[H]
    \centering
    \includegraphics[width=0.8\textwidth]{exercise03/exercise03-5.png}
    \caption{Annäherung der Ableitung einer Funktion - Schritt 5: h = 0.01}
    \label{fig:exercise03-5}
\end{figure}

%----------------  Bibliography   ------------------------------------------
%\clearpage
%\printbibliography

%----------------  List of figures   ---------------------------------------
\ifnum\totvalue{figures}>0
  \clearpage
  \listoffigures
\fi
 
%----------------  List of tables   ----------------------------------------
\ifnum\totvalue{tables}>0
  \clearpage
  \listoftables
\fi

%----------------  List of listings   --------------------------------------
%\clearpage
%\lstlistoflistings 

%----------------  Glossary and acronyms  ----------------------------------
%\clearpage
%\printglossary[type=main]
%\printglossary[type=acronym]

%------------ Index ----------------------
%\clearpage
%\printindex

%------------ Appendix ----------------	
%\appendix

\end{document}
